\documentclass[a4paper]{article}
\usepackage[utf8]{inputenc}
\usepackage[T1]{fontenc}
\usepackage{graphicx}
\usepackage{hyperref}
\usepackage{lmodern}
\usepackage{amsmath}
\usepackage[backend=biber]{biblatex}
\addbibresource{citiranja.bib}

\newcommand{\autor}{ \centering Nenad Vuletić, Vladimir Krsmanović, Marko Polić}
\newcommand{\naslov}{Remzijeva teorija na hipergrafovima}
\newcommand{\datum}{Novi Sad, 2020}
\newcommand{\mentor}{dr. Maja Peh}
\begin{document}
	\begin{center}
		
		\includegraphics[width=2cm]{grbPMF}\hfill
		\parbox[b]{45ex}{\centering 
			Univerzitet u Novom Sadu\\
			Prirodno-matematički fakultet\\
			Departman za matematiku i informatiku}\hfill 
		\includegraphics[width=2cm]{grbUNS}
		
		\vspace{22ex}
				
		{\Huge {\bf \setlength{\baselineskip}{1.5\baselineskip}\naslov}}
		
		\vspace{4ex}
		Seminarski rad
		
		\vspace{12ex}
		\parbox[b]{\textwidth}{{\Large {\bf \hspace{1cm}\autor}}}
		\vspace{10ex}
		
		{\Large Mentor:}
		
		{\Large  \textbf{\mentor}}
		
		\vfill
		
		\datum
		
	\end{center}
	\thispagestyle{empty}
	\newpage
	
	\tableofcontents
	\newpage
	\section{Uvod}
	Ubaciti Polićev tekst
	\section{Postavka i dokaz Remzijeve teorije za hipergrafovima}
	\subsection{Postavka problema}
	\subsection{Dokaz Remzijeve teoreme za hipergrafove}
	U ovom poglavlju dokazujemo Remzijevu teoriju %preformulisati 
	za hipergraf $ G_{k}(V, E)$ gde je V skup čvoreva, a E je skup skupova k-tog stepena čiji su elementi čvorovi iz skupa V.
	
	\noindent Treba da pokažemo $m_{3} + 1 \rightarrow (L_1, L_2)^k$. 
	Matematičkom indukcijom dokazujemo:
	\begin{equation}
		m_{3} + 1 \rightarrow (L_1, L_2)^k		
	\end{equation}
	\begin{description}
		\item \textbf{Indukcijska baza} \\
		Radimo indukciju nad elementom k, i iz Remzijeve teoreme za obične grafove (gde je k = 2), dobijamo indukcijsku bazu. 
		\item \textbf{Indukcijska hipoteza}\\
		Iz indukcijske hipoteze dobijamo da važi
		\begin{equation}\label{m1}
		m_{1} \rightarrow (L_1 - 1, L_2)^k		
		\end{equation}
		\begin{equation}\label{m2}
		m_{2} \rightarrow (L_1, L_2 -1)^k		
		\end{equation}
		\begin{equation}\label{m3}
		m_{3} \rightarrow (L_1, L_2)^{k-1}		
		\end{equation}
		\item \textbf{Indukcijski korak}\\	
		Konstruišimo prvo k-uniformni hipergraf P sa $n = m_3 + 1$ čvorova. Nad tim hipergrafom je definisano bojenje $\chi$ takvo da boji sve k-elementne hipergrane hipergrafa P ili u boju 1 ili u boju 2.\begin{equation}
			\chi : [ \underline{n} ]^k  \rightarrow \underline{2}
		\end{equation} \\
		Bez umanjenja opštosti onda možemo uočiti proizvoljan čvor a unutar skupa čvoreva hipergrafa P, i onda možemo uočiti hipergraf $P^\prime$ čiji je skup čvorova $[\underline{n} \textbackslash \{ a \}]$, i njegova kardinalnost je stepena $m_3$. Po posledici \ref{m3} znamo da postoji k-1 uniformni hipergraf sa $m_3$ čvorova. %preformulisati
		\\
		Na njemu onda definišemo bojenje $\overline{\chi}$ takvo da boji hipergrane stepena k-1 hipergrafa $P^\prime \textbackslash \{ a \}$ u istu boju kao što ih boji  $\chi$ u $P^\prime$ dok je imalo a.
		\begin{equation}
			\overline{\chi} : [ \underline{n} \textbackslash \{a\} ]^{k-1}  \rightarrow \underline{2}
		\end{equation} %preforumilasti
		Takodje definišemo preslikavanje M koje uzima sve hipergrane koje su imale a, i boji ih u istu boju kao $\chi$ dok su imale a.
		\begin{equation}
			M \rightarrow \chi (M \cup \{a\})
		\end{equation}
		\\
		Iz posledice \ref{m3} takodje sledi da postoji $m_1$ čvorova gde sve k-1 elementne hipergrane nad njima su obojene bojom 1 (po bojenju $\overline{\chi}$)
		Taj podskup čvorova nazivamo $K_1$, i njegova kardinalnost je  $m_1$. \footnote{Postupak je analogan za grane obojene bojom 2, gde nastaje podskup $K_2$. Analogni postupak je ispisan u appendixu}. \\
		Posmatrajmo sada ponovo bojenje $\chi$. Konstruišimo sada restrikciju $\chi$ nad  skupom čvoreva $K_1$.
		\begin{equation}
			\chi_1 : [ K_1 ]^k  \rightarrow \underline{2}
		\end{equation}\\
		Iz posledice \ref{m1} sledi da na hipergrafu koji sadrži $m_1$ čvorova postoji skup k-elementnih hipergrana obojenih bojom 1. Neka skup koji sadrži te grane bude $T_1$, i po posledici \ref{m1} znamo da je njegova kardinalnost jednaka sa $L_1 - 1$. Analgonim postupkom dolazimo do  skupa k-elementnih hipergrana boje 2 koji ćemo obeležiti sa $T_2$, i njegova kardinalnost (ponovo po posledici \ref{m1}) je jednaka $L_2$. \\
		Skupu $T_1$ sada možemo pridružiti čvor a, i tim proširenjem ponovo posmatramo k-elementne hipergrane nad novim skupom čvorova $T = T_1 \cup \{a\}$.\\
		Posmatrajmo sada sve (k-elementne) grane tog skupa.\\
		One mogu ili da 
		\begin{enumerate}
			\item \textbf{Sadrže a}\\
				U tom slučaju se onda sve te grane boje isto kao što ih boji $\overline{\chi}$ kada ne bi imale čvor a. Pošto $\overline{\chi}$ boji sve k-1 elementne hipergrane $K_1$ u boju 1 . A pošto je $T_1$ podskup, onda dobijamo da 
				\begin{equation}
					[ T_1 ]^{k-1} \subseteq \chi^\prime (1)
				\end{equation}
				Potom možemo uključiti čvor a, i bojenje $\chi$ će ih isto bojiti u boju 1.
				\begin{equation}
					[ T_1 \cup \{a\} ]^{k} \subseteq \chi^\prime (1)
				\end{equation}
				Odatle dolazimo do zaključka 
				\begin{equation}
					[ T ]^{1} \subseteq \chi^\prime (1)
				\end{equation}
			\item  \textbf{Ne sadrže a} \\
				Onda sve k-elementne hipergrane$T_1$ su već boje 1, time direktno dobijamo
				\begin{equation}
					[ T ]^k \subseteq \chi^\prime (1)
				\end{equation}
		\end{enumerate}
		Ovime smo pokazali da se sve k-elementne hipergrane iz T boje bojom 1. Time zaključujemo da postoji potpun podhiperhgraf koji sadrži $L_1$ čvorova, i da su sve njegove hipergrane obojene bojom 1. Time ispunjavamo uslov za $L_1$ iz početne pretpostavke $m_3 + 1 \rightarrow (L_1, L_2)^k$.\footnote{Analognim postpukom za $K_2$ ispunjavamo uslov uz $L_2$. Analogni postpuak je ispisan u appendixu}\\
		{\large Q.E.D.}
	\end{description}
	\subsection{Dokaz Remzijeve teoreme za polihromatske hipergrafove}
	\section{Pregled stanja Remzijeve teorije nad hipergrafovima}
	Pošto su i polja grafova i Remzijeve teorije oba poprilično mlada, posebno u poredjenju sa ustaljenijim matematičkim disciplinama, još uvek je slabo istražena. I to posebno u poljima hipergrafova.\\ Većina radova na ovom polju se bavi primarno sa uniformnim hipergrafovima, upravo zbog kompleksnosti rada se neunifomrnim hipergrafovima.\\ Kao posledica težine izučavajna Remzijeve teorije, matematičari su pribegli generelizacijama problema i kompjuteskim simulacijama. Iako su oba van osnovne teme ovog rada, o kompjuteskim simulacijama i jedinom poznatom remzijevom broju za hipergrafove ćemo diskutovati u sledećem poglavlju, dok ćemo o generelizacijama problema pričati u kasnijem poglavlju (Gallai - Remzijevim bojenjima).
	\subsection{Poznati remzijev broj za hipergrafove}%opraviti citiranja%
	Trenutno jedini poznati remzijev broj za hipergrafove (u klasičnom smislu, posmatrajući potpune indukovane podhipergrafove)\cite{pregled}%proveriti i citirati
	 je dobijen uz pomoć računara. Brendan D. McKay i Stanislaw P. Radziszowski, su 1991 godine objavili svoj rad \textcite{HGremzibroj} gde su odredili broj tačaka potreban da se bihromatskim bojenje trouglova nad skupom od 13 tačaka mora uvek mora pronaći monohromatski četvorougao.%preformulisati
	\begin{equation}
		R(4, 4; 3) = 13
	\end{equation}
	Na osnovu već poznatih granica ($R(4, 4; 3) \leq 15$ Giraud))%citiraj Girauda, ima link u HGrezmi radu, 
	, kao i veze izmedju Tur\'{a}novih sistema\footnote{Donekle slični remzijevim brojevima, Tur\'{a}novi brojevi predstavljaju minimalni broj grana koje  ispunjavaju uslov da onda za bilo koji podskup čvoreva tog hipergrafa uvek sadrži bar jendu granu. Ali pošto je to van opsega ovog rada, nećemo ulaziti dublje u teoriju Tur\'{a}novih sistema } i Remzijevih brojeva, McKay i Radziszovski su uspeli prvo da spuste granicu na $R(4, 4; 3) \leq 13$, a onda i da pomoću kompjutera pronadju taj graf, kroz proces u kojem su generisali preko 200,000 grafova. Ovaj problem možemo loako posmatrati kroz prizmu hipergrafova (slično kao i kod Erd\"{o}s-Szekeres problema), gde tražimo koliko je potrebno čvorova za 3-uniformni hipergraf tako da on uvek ima ili potpun podhipergraf indukovan sa 4 čvora, ili 4  nezavisna čvora. %citirati!! citirati i preformulisati footnote kod Tur\'{a}novih bsistema, i citirati !!! potencijalno ubaciti sliku tog grafa
	
	\subsection{Remzijeva teorija nad stablima}
	\subsection{Gallai-Remzi brojevi}
	\section{Bibliografija}	
	\printbibliography
\end{document}